\documentclass[a4paper,11pt]{article}
\usepackage{listings}
\usepackage{scalefnt}
\usepackage{cancel}
\usepackage{picinpar}
\usepackage{subfig}
\usepackage{graphicx}
\usepackage{amssymb}
\usepackage{amsmath}
\usepackage[brazilian]{babel}
\usepackage[utf8]{inputenc}
\usepackage[T1]{fontenc}
\usepackage{setspace}
\usepackage[left=1.5cm,right=1.5cm,top=2.0cm,bottom=1.5cm]{geometry}
\date{\today}
\title{Métodos Computationais em Física 2020\\ Pêndulo Amortecido Forçado} %Coloque o título
\author{João Henrique de Sant'Ana  NUSP 9783362} % Coloque o seu nome e NºUSP

% – inicio codigo fonte para inserir o script de colocar o seu arquivo texto, não mecher aqui
\usepackage{listings}

\lstset{numbers=left,
stepnumber=1,
firstnumber=1,
numberstyle=\tiny,
extendedchars=true, %				Não mecha!
breaklines=true,
frame=tb,
basicstyle=\footnotesize,
stringstyle=\ttfamily,
showstringspaces=false
}
\renewcommand{\lstlistingname}{Code}
\renewcommand{\lstlistlistingname}{Lista de Listagens}
% – fim codigo fonte

\begin{document}  

\maketitle
\section*{Introdução}
Nesse manual, tenho o intuito de instruir o usuário a usar os script's que resolvem o problema do pêndulo amortecido forçado. Primeiro de tudo você precisa ter python3 instalado em sua máquina e também ter os módulos \textit{numpy}, \textit{matplotlib} e \textit{scipy}. Se você estiver numa distribuição linux ou MacOS, você precisa apenas abrir o terminal e ir no diretório onde estão os aquivos $.py$ e dar o seguindo comando python3 $nome\_arquivo.py$. Se voce estiver num ambiente de programação é so abrir o aquivo $.py$ e rodar. Não tem erro. Os arquivos são
\begin{itemize}
	\item $espaco\_fase.py$
	\item $mapa\_poincare.py$
	\item $diagrama\_bifurcacao.py$
	\item $expoente\_lyapunov.py$
	\item $energia\_oscilador.py$
\end{itemize}
\section*{Equações}
 \begin{equation}\label{eq:equação_pêndulo_numerica}
        \begin{cases}
        &\dot{\omega}= -\frac{1}{q}\dot{\theta} -sin(\theta) + Fsin(\Omega_{d}t)\\
        &\dot{\theta}=\omega
        \end{cases}
    \end{equation}  
Para resolver numericamente utilizamos o algoritmo de runge-kutta de 4 ordem. Em todos os aquivos utilizamos a seguinte notação:
\begin{equation}\label{eq:notação}
        \begin{cases}
        & w = \Omega_{d}\\
        & y = \theta\\
        & z = \dot{\theta}
        \end{cases}
\end{equation}
Em todos os programas utilizamos a mesma função rungekutta$(y0,z0,F,q,w)$, onde $y0=\theta(0)$, $z0=\dot{\theta(0)}$ que retorna a lista tempo, y e z, além de que as variáveis h e N, intervalo do tempo e número total de elementos na lista do tempo, são variáveis globais. Em toda análise da dinâmica usamos $0<F\leq1.5$, $q=2$ ou $q=4$ e $w=2/3$. Como o algoritmo de runge-kutta é relativamente simples, resolvi deixar a mesma função em todos os arquivos para você rodar diretamente, sem erro. Qualquer dúvida é so me enviar um e-mail\footnote{joao.henrique.santana@usp.br}. Bom proveito!
\section*{Comentario}
A energia do oscilador utilizamos a expressão 
\begin{equation}
E = ml^2\dot{\theta}^{2} + mgl(1-cos(\theta))
\end{equation}
Para o plot da energia do oscilador, divimos tudo por $ml^{2}$ e ajustamos $\frac{g}{l} = 1$, daí
\begin{equation}
 E' = \dot{\theta}^{2} + (1-cos(\theta))
\end{equation}
\section*{Código}
Para você ler código e os comentários de uma forma mais confortável, resolvi colocar nesse manual. Para não torna repetitivo só comentei a rungekutta no primeiro código. 
\lstinputlisting[language=python, label=sqlselect, caption={Espaço de Fase}]{espaco_fase.py}
\lstinputlisting[language=python, label=sqlselect, caption={Mapa de Poincaré}]{mapa_poincare.py}
\lstinputlisting[language=python, label=sqlselect, caption={Diagrama de Bifurcação}]{diagrama_bifurcacao.py}
\lstinputlisting[language=python, label=sqlselect, caption={Exponte de Lyapunov}]{expoente_lyapunov.py}
\lstinputlisting[language=python, label=sqlselect, caption={Energia do Oscilador}]{energia_oscilador.py}
\end{document}